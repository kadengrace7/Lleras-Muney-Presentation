\documentclass{beamer}
\usepackage{booktabs}
\usepackage{fancyhdr}
\usepackage{extramarks}
\usepackage{amsmath}
\usepackage{amsthm}
\usepackage{amsfonts}
\usepackage{tikz}
\usepackage[plain]{algorithm}
\usepackage{algpseudocode}
\usepackage{amssymb}
\usepackage{graphicx}
\usepackage{color}
\usepackage{wrapfig}
\usepackage{subfig}
\usepackage{caption}
\usepackage{lipsum}
\usepackage{dcolumn}
\usepackage{mathptmx}
\usepackage{tikz}
\usetikzlibrary{fit, tikzmark}
\usetheme{utk}

\AtBeginSection[]
 {
 \addtocounter{framenumber}{-1}
    \begin{frame}
        \tableofcontents[currentsection]
    \end{frame}
 }

\title[Army Relocation \& Air Pollution] % short title
{\small The Needs of the Army: Using Compulsory Relocation in the Military to Estimate the Effect of Air Pollutants on Children's Health} % full title
\author[Grace] % short names
{Adriana Lleras-Muney, JHR 2010} % full names

\date{Presented December 4, 2024, by Kaden Grace}

\institute{}

\begin{document}

\begin{frame}
\titlepage
\end{frame}


\begin{frame}{Research Question}
    \begin{itemize}
        \item What is the effect of air pollution on respiratory hospitalizations for children?
    \end{itemize}
\end{frame}

\begin{frame}{Contributions}
    \begin{itemize}
        \item Identification: leverages random assignment to treatment
        \item Outcome: hospitalizations instead of mortality
        \item Method: IDW vs. Kriging, monitor measurement error
    \end{itemize}
\end{frame}

\begin{frame}{Military Relocation}
    \begin{itemize}
        \item Random reassignment at least every three years, no more than once per year (avg: 2.5 yrs)
        \item Over 20 years, individuals relocate an average of 12 times
        \item Conditional on rank and military occupation
        \item This paper shows evidence that randomization is successful 
    \end{itemize}
\end{frame}

\begin{frame}{Identification}
    \begin{itemize}
        \item Air pollution differs across bases and over time
        \item Military children aged 0-5 are exposed, and some may be hospitalized
        \item Exposure is dependent on the parent's assignment
    \end{itemize}
\end{frame}

\begin{frame}{Data}
\begin{itemize}
    \item Family location assignments and demographics from the Defense Manpower Data Center (DMDC)
    \item Hospitalizations at Military Treatment Facilities 
    \item Hospitalizations at other hospitals from Champus/Tricare
    \item Pollution from the EPA
    \item Weather data from the National Climactic Center
\end{itemize}
\end{frame}

\begin{frame}{Results}
    \begin{itemize}
        \item Significant results only for ozone among children aged 2-5
        \begin{itemize}
            \item 15\% increase in ozone $\implies$ 8-23\% increase in prob. of respiratory hospitalization
        \end{itemize}
        \item Moving children to low-pollution areas $\implies$ 143,100 fewer hospitalizations, 480,000 fewer ER visits, and 3.2 million fewer physician visits per year
        \item \$928m savings in direct medical expenditure
    \end{itemize}
\end{frame}




\end{document}